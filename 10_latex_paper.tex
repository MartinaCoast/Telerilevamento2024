\documentclass[12pt]{article} %spiega che tipo di documento realizziamo
\usepackage{graphicx} %pacchetto per inserire le immagini
\usepackage{hyperref} %pacchetto per creare un collegamento ipertestuale
\usepackage{natbib} %pacchetto per fare la bibliografia
\usepackage{lineno} %pacchetto per visualizzare i numeri di riga
\linenumbers %questo non ho capito a cosa serve


\title{My first LaTeX doc} %titolo del documento
\author{Martina Costa}
% \date{May 2024} %data del nostro documento, per rimuoverla o tolgo il contenuto tra parentesi o la mantengo come commmento

\begin{document}

\maketitle %prende tutti i pezzi scritti sopra e li mette all'inizio del documento

\begin{abstract} % per inserire l'abstract
    Il nome generico (Helianthus) deriva da due parole greche: ”helios” (= sole) e ”anthos” (= fiore) in riferimento alla tendenza della pianta a girare sempre il bocciolo verso il sole, 
    prima della fioritura (il fiore maturo invece è sempre rivolto ad est). Questo comportamento è noto come eliotropismo.

L'epiteto specifico (annuus) indica il tipo di ciclo biologico (annuale). Anche il nome comune italiano (Girasole) richiama la rotazione dei boccioli in direzione del sole. 
Il termine “girasole” è anche usato per indicare le altre piante appartenenti al genere "Helianthus", molte delle quali sono perenni.
\end{abstract}

\tableofcontents %funzione per inserire il sommario

\section{Introduction} \label{sec:intro} %cosi gli dico che introduzione è una sezione! se sbagliassi a dare il nome alla label quando ricompilo mi mette ?? perchè non ho dato lo stesso nome a label e ref in fondo nella discussione
\textbf{Hello darkness, my old friend} %per mettere il testo in grassetto
\textit{I've come to talk with you again} %per mettere il testo in corsivo it = italic
Because a vision softly creeping
Left its seeds while I was sleeping
And the vision that was planted in my brain
Still remains
Within the sound of silence.

\bigskip % per lasciare una riga vuota grande OPPURE doppio backslash \\

In restless dreams I walked alone
Narrow streets of cobblestone
'Neath the halo of a street lamp
I turned my collar to the cold and damp
When my eyes were stabbed by the flash of a neon light
That split the night
And touched the sound of silence.

\smallskip % lascia sempre una riga vuota ma un po' più stretta

And in the naked light I saw
Ten thousand people, maybe more
People talking without speaking
People hearing without listening
People writing songs that voices never share
No one dared
Disturb the sound of silence.

In this thesis we deal with love in fourth dimension, one of the main themes in life \cite{Berscheid2010}. We are very intereste in space shuttle \citep{Paul2013}.

% \citep{Paul2013} se voglio mettere tutta la citazione tra parentesi, 
% \citet{Paul2013} mette tra parentesi solo l'anno
% \cite{Paul2013} di default mi mette tra parentesi l'anno

%NUOVA SEZIONE DEI METODI
\section{Methods}
\subsection{Srudy area}
\subsection{Algorithms}

%per scrivere un'EQUAZIONE:
The first equation used was Equation \ref{eq:sum}:
\begin{equation} %inizio dell'eq
    T = \sum p_i
    \label{eq:sum} %attribuiamo un'etichetta all'eq per fare riferimenti
\end{equation} %fine dell'eq

\noindent In this thesis we made use of Equation \ref{eq:newton}:
\begin{equation}
    F = \sqrt[3]{G \frac{m_1 \times m_2}{d^2}} %\sqrt per mettere sotto radice
    \label{eq:newton}
\end{equation}


\section{Results}
Our results are in line with previous papers, introduced in section \ref{sec:intro}. % cosi visto che prima avevamo dato un'etichetta all'introduzione adesso mi basta usare il ref e quando ricompilo mi aggiunge il colelgamento con la sezione

In this thesis one of the plant used was Figure \ref{sun}.
\footnote{sunflower from italy} %note a pie di pagina

\section{Discussion}
% ELENCO PUNTATO
\begin{itemize}
    \item love is important
    \item expecially in the fouth dimension
    \item ecc ecc
\end{itemize}

%ELENCO NUMERATO
\begin{enumerate}
    \item  ciao
    \item giovedi
    \item albero
\end{enumerate}

\section{Data availability}
All the data used in this thesis are available at:
\url{https://www.nasa.gov/} % funzione per inserire un link

\newpage % per inserire una nuova pagina
\begin{figure}
    \centering %per centrare l'immagine nella pagina
    \includegraphics[width=\textwidth]{sun.jpg}  %così l'immagine viene larga quanto il testo. Se volessimo farlo la metà, basterebbe scrivere width=0.5
    \caption{Helianthus annuus} %didascalia dell'immagine
    \label{sun}
\end{figure}

%inserire bibliografia
\begin{thebibliography}{999} %questo è un tipo di bibliografia
\bibitem[Berscheid, 2010]{Berscheid2010} % quella tra graffe è la label o etichetta, quella tra quadre è come verrà visualizzata nel testo
    Berscheid, E. (2010). Love in the fourth dimension. Annual review of psychology, 61, 1-25. %poi sotto copio la citazione come la metterei nell'elenco

\bibitem[Paul, 2013]{Paul2013}
Paul, A. L., Wheeler, R. M., Levine, H. G., \& Ferl, R. J. (2013). Fundamental plant biology enabled by the space shuttle. American journal of botany, 100(1), 226-234.
\end{thebibliography} 
%ho messo un \ prima della e commerciale & perchè per latex è un simbolo, con la \ lo proteggo e non glielo faccio leggere come simbolo

\newpage
\bigskip
% per inserire un box con altre info che mi potrebbero servire
\hline
\textbf{Box 1 - ciao}
\bigskip
\hline %disegna una linea orizzontale

\begin{itemize}
\item hola
\item fiore
\item roccia
\end{itemize}
\bigskip
\hline

\end{document}
