\documentclass{beamer} % beamer è il tipo di domcumento per la presentazione
\usepackage{graphicx} % Necessario per inserire immagini
\usepackage{listings} % Pacchetto utile per caricare algoritmi direttamente dagli script
\usepackage{multicol} % Per visualizzare gli oggetti su due colonne
\usepackage{url} % Per inserire pagine web

% sito per info cambio tema: https://mpetroff.net/files/beamer-theme-matrix/
\usetheme{Frankfurt}
\usecolortheme{seahorse}

\title{\textbf{Il crollo del Birchgletscher sul paese di Blatten (Lotschen, Svizzera)}}
\subtitle{Analisi dell'impatto causato dalla frana}
\author{Martina Costa}
\date{17 settembre 2025}

\begin{document}

\maketitle

\AtBeginSection[] % Do nothing for \section*
{
\begin{frame}{Indice}
\tableofcontents[currentsection]
\end{frame}
}


\section{Il crollo}

\begin{frame}{La località}
\begin{figure}
    \centering
    \includegraphics[width=0.65\linewidth]{frana.jpg}
    \vspace{-10pt}
    \caption{Area d'indagine}
    \label{fig:placeholder}
\end{figure}
\vspace{-20pt}
    {\small
    \begin{itemize}
        \item Valle Lötschental, Canton Vallese, Svizzera
        \item Fiume Lonza
        \item 1540 m s.l.m
        \item Monitoraggio a seguito di frane negli anni '90
    \end{itemize}
    }
\end{frame}

\begin{frame}{Timeline}
\begin{multicols}{2}
\begin{figure}
    \centering
    \includegraphics[width=0.9\linewidth]{mucca3.jpg}
    \label{fig:placeholder}
\end{figure}
\columnbreak
    {\small
    \begin{itemize}
        \item \textbf{15 maggio}: si stacca un pezzo del monte Nesthorn, formando una frana che travolge il ghiacciaio e si ferma a 500m dal fiume
        \item \textbf{17 maggio}: evacuazione del paese di Blatten
        \item Giorni successivi: continuano a cadere massi e detriti dal Nesthorn, accumulandosi sopra al ghiacciaio
        \item \textbf{27 maggio}: primo crollo parziale
        \item \textbf{28 maggio}: il ghiacciaio e i detriti sommergono il paese
    \end{itemize}
    }
\end{multicols}
\end{frame}

\begin{frame}{Il lago}
    \begin{figure}
        \centering
        \includegraphics[width=0.8\linewidth]{lago1.jpg}
        \caption{Formazione di un lago a seguito dello sbarramento del fiume Lonza e dell'accumulo di ghiaccio.}
        \label{fig:placeholder}
    \end{figure}
\end{frame}

\begin{frame}{Obbiettivo del progetto}
Lo studio si pone come obbiettivo quello di quantificare le modificazioni del territorio a seguito del crollo del ghiacciaio.


\bigskip Per avere copertura nivale e situazione vegetazionale simili, sono state selezionate le immagini relative al 23 agosto 2024 e 25 agosto 2025.

\bigskip L'area analizzata ha un'estensione di 50.96 km$^{2}$.
    
\end{frame}


\section{Raccolta dati}
 \begin{frame}{Raccolta dati}
        Le immagini utilizzate per questo progetto sono state catturate dal satellite Sentinel-2 e scaricate dal Copenicus Browser.

        \bigskip Per avere immagini nitide è stata impostata una copertura di nuvole al 5\%.
            
        Sono state scaricate le bande 2, 3, 4 e 8 in formato .TIFF a 16 bit.

        \bigskip Pacchetti utilizzati:
        \begin{itemize}
            \item    \texttt{library(terra)} %per font Courier
            \item    \texttt{library(imageRy)} 
            \item    \texttt{library(viridis)}
            \item    \texttt{library(ggplot2)} 
            \item    \texttt{library(patchwork)}
        \end{itemize}    
        
\end{frame}

\begin{frame}{Codice}
    Una volta scaricate le varie bande, queste sono state sovrapposte per creare le immagini in True Color.

    \bigskip In seguito è stato sostituito il \textbf{NIR} al \textbf{rosso}, mettendo in risalto sia la vegetazione che la formazione del lago, in quanto il vicino infrarosso viene riflesso dalla vegetazione e assorbito dagli specchi d'acqua.

    \bigskip \texttt{par(mfrow=c(1,2))}
    
    \texttt{im.plotRGB(G24, 4,2,1)}
    
    \texttt{title("2024 (nir)", line=-2)}
    
    \texttt{im.plotRGB(G25, 4,2,1)}
    
    \texttt{title("2025 (nir)", line=-2)}
\end{frame}

\begin{frame}{NIR on Red}
    \begin{figure}
            \centering
            \includegraphics[trim=0cm 5cm 0cm 2cm, clip, width=0.8\linewidth]{NIR_confronto.jpeg}
            \caption{False Color}
            \label{fig:placeholder}
        
    \end{figure}
\end{frame}

\begin{frame}{NDVI}
    Calcolo l'NDVI (Normalized Difference Vegetation Index) seguendo la formula:
    \begin{equation}
                NDVI = \frac{(NIR - RED)}{(NIR + RED)}
    \end{equation}

    \bigskip Questo indice è un indicatore della presenza di vegetazione, in quanto la vegetazione assorbe la luce nel rosso visibile e riflette fortemente la luce nel vicino infrarosso.

    \begin{center}
        \bigskip \texttt{NDVI\_24 = (G24[[4]]-G24[[3]])/(G24[[4]]+G24[[3]])}
        
        \bigskip \texttt{NDVI\_25 = (G25[[4]]-G25[[3]])/(G25[[4]]+G25[[3]])} 
    \end{center}
\end{frame}

\begin{frame}{Classificazione in base all'NDVI}
    \begin{figure}
        \centering
        \includegraphics[trim=0cm 6cm 0cm 5cm, clip, width=1\linewidth]{NDVI_confronto.jpeg}
        \caption{Visualizzazione delle immagini elaborate attraverso l'NDVI. La colorazione scelta è viridis in modo da evidenziare la vegetazione in giallo mentre il resto del territorio apparirà in una scala di blu.}
        \label{fig:placeholder}
    \end{figure}
\end{frame}

\begin{frame}{NDWI}
    Calcolo poi l'NDWI (Normalized Difference Water Index) seguendo la formula:
    \begin{equation}
                NDWI = \frac{(GREEN - NIR)}{(GREEN + NIR)}
    \end{equation}
    
    \bigskip Questo indice sfrutta il fatto che l'acqua assorbe fortemente il NIR e riflette il verde, consentendo di evidenziare il cambiamento nel contenuto di acqua superficiale della valle.

    \begin{center}
        \bigskip \texttt{NDWI\_24 = (G24[[2]]-G24[[4]])/(G24[[2]]+G24[[4]])}
    
        \bigskip\texttt{NDWI\_25 = (G25[[2]]-G25[[4]])/(G25[[2]]+G25[[4]])}
    \end{center}
\end{frame}

\begin{frame}{Classificazione in base all'NDWI}
\begin{figure}
    \centering
    \includegraphics[trim=0cm 6cm 0cm 5cm, clip, width=1\linewidth]{NDWI_confronto.jpeg}
    \caption{Visualizzazione delle immagini elaborate attraverso l'NDWI. In questo caso la colorazione scelta è cividis, in modo da evidenziare in giallo l'acqua.}
    \label{fig:placeholder}
\end{figure}
    
\end{frame}

\begin{frame}{Calcolo delle classi}
    I risultati ottenuti con il calcolo di NDVI e NDWI per entrambi gli anni sono stati suddivisi in 2 cluster, calcolando poi la percenutale di copertura per ogni cluster.

    \begin{center}
        \bigskip \texttt{cG24 <- im.classify(NDVI\_24, num\_clusters=2)}
    
        \texttt{cG25 <- im.classify(NDVI\_25, num\_clusters=2)}

        \bigskip \texttt{wG24 <- im.classify(NDWI\_24, num\_clusters=2)}

        \texttt{wG25 <- im.classify(NDWI\_25, num\_clusters=2)}
    \end{center}
    
    \bigskip\textnormal{Cluster per NDVI: vegetazione, altro (suolo nudo/neve/acqua)}
    
    \textnormal{Cluster per NDWI: acqua, altro (vegetazione/rocce)}
\end{frame}

\begin{frame}{Creazione dei dataframe}
    Esempio di calcolo delle percentuali:

    \texttt{f24 <- freq(cG24)}

    \texttt{tot24 <- ncell(cG24)}

    \texttt{prop24 = f24 / tot24}

    \texttt{perc24 = prop24 * 100}
    
    \texttt{perc24}
    
    \bigskip Le percentuali sono state utilizzate per creare due Dataframe, uno per anno, e ognuno con la rispettiva percentuale di vegetazione e acqua:

    \begin{figure}
        \centering
        \includegraphics[width=0.45\textwidth]{1tab2024.png}
        \includegraphics[width=0.45\textwidth]{2tab2025.png}
    \end{figure}
\end{frame}

\begin{frame}{Esempio di classificazione}
    \begin{figure}
        \centering
        \includegraphics[trim=0cm 6cm 0cm 6cm, clip, width=0.7\linewidth]{classNDVI.jpeg}
        \caption{Classificazione NDVI}
        \label{fig:placeholder}
    \end{figure}
    \begin{figure}
        \centering
        \includegraphics[trim=0cm 6cm 0cm 6cm, clip, width=0.7\linewidth]{classNDWI.jpeg}
        \caption{Classificazione NDWI}
        \label{fig:placeholder}
    \end{figure}
\end{frame}

\begin{frame}{Relazione tra NDVI e NDWI}
    Plottando la classificazione non si coglie molto la differenza tra i 2 cluster calcolati usando l'NDVI e i 2 cluster ottenuti usando l'NDWI, in quanto le percentuali hanno valori simili.

    \bigskip Quindi, per visualizzare meglio i cambiamenti e per avere nello stesso plot i due indici, questi possono essere messi in relazione calcolando la loro differenza per entrambi gli anni, secondo la formula:
    \begin{equation}
        Diff = NDVI - NDWI
    \end{equation}

    \bigskip In questo caso il range andrà da +2 a -2 perchè entrambi gli indici sono normalizzati compresi tra +1 e -1.
\end{frame}

\begin{frame}{Differenza NDVI-NDWI}
    A valori positivi corrisponde dominanza di vegetazione e a valori negativi corrisponde una dominanza di acqua.
    
    \bigskip Esempio: caso estremo in cui vegetazione è +1 e acqua assente quindi -1, diff(NDVI-NDWI) = +1 - (-1) = +2

    \begin{figure}
        \centering
        \includegraphics[trim=0cm 6cm 0cm 6cm, clip, width=0.9\linewidth]{diffNDVI-NDWI.jpeg}
        \caption{Differenza NDVI - NDWI. La colorazione usata è plasma.}
        \label{fig:placeholder}
    \end{figure}
\end{frame}

\section{Risultati}
\begin{frame}{Grafico a barre}
    \begin{figure}
        \centering
        \includegraphics[width=0.8\linewidth]{grafico_barre.jpeg}
        \label{fig:placeholder}
    \end{figure}
    \begin{center}
        \vspace{-0.5cm}Valori di vegetazione e acqua nell'area di indagine rispettivamente in agosto 2024 e agosto 2025.
    \end{center}
\end{frame}

\section{Conclusioni}
\begin{frame}{Conclusioni}
    \begin{itemize}
        \item I risultati ottenuti mediante le analisi su R risultano rappresentativi dell'impatto che il crollo del ghiacciaio ha avuto sulla morfologia della valle;
        \bigskip
        \pause \item Dal calcolo delle percentuali, si può osservare come la frana abbia portato ad una diminuzione circa del 5\% della vegetazione;
       \bigskip
       \pause \item Mentre l'ostruzione del fiume e la conseguente formazione del lago hanno portato ad aumento di acqua superficiale nell'area circa del 4\%.
    \end{itemize}
\end{frame}


\section{Links}

\begin{frame}{Github e fonti}
Il mio account \textbf{Github:}
    \url{https://github.com/MartinaCoast}

\bigskip\textbf{Sitografia:}
\begin{itemize}
    \item\url{https://www.rsi.ch/}
    \item\url{https://www.rsi.ch/info/svizzera/Blatten-quando-la-montagna-crollò--2875870.html}
    \item\url{https://it.wikipedia.org/wiki/Blatten}
    
\end{itemize}
\end{frame}

\begin{frame}
    
    \begin{multicols}{2}
    \begin{figure}
        \centering
        \includegraphics[width=0.9\linewidth]{cono detrito.jpeg}
        \label{fig:placeholder}
    \end{figure}
    \columnbreak

     In immagine: Il cono di detrito causato dalla frana, visto dal versante opposto. Si può intravedere il lago.

    \bigskip
    \bigskip
    \begin{center}
        \Large {Grazie dell'attenzione!}
    \end{center}
    \end{multicols}
      
\end{frame}

\end{document}
