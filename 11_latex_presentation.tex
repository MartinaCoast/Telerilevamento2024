% COME CREARE UNA PRESENTAZIONE IN LATEX

\documentclass{beamer} % beamer è il tipo di domcumento per la presentazione
\usepackage{graphicx} % Required for inserting images

% per cambiare colore e tema della presentazione vai qui e cerca il tuo preferito: https://mpetroff.net/files/beamer-theme-matrix/
\usetheme{Frankfurt} % per cambiare tema
\usecolortheme{crane} % per cambiare colore di tema

\title{My first presentation}
\author{Martina Costa}
%\date{May 2024}

\begin{document}

\maketitle

% inseriamo un sommario delle slide
\AtBeginSection[] % Do nothing for \section*
{
\begin{frame}{Outline}
\tableofcontents[currentsection]
\end{frame}
}

\section{Introduction}

% aggiungere una diapositiva "frame"  
\begin{frame}{My firts slide}
    My first slide
\end{frame}

% ELENCO PUNTATO
\begin{frame}{Itemization}
    \begin{itemize}
        \item My first sentence
        \item \pause My second one % \pause serve per inserire una sorta di animazione e mi fa comparire il secondo punto in un secondo momento
        \item \pause My third one here
    \end{itemize}
\end{frame}

\begin{frame}{My second slide}
    My second slide
\end{frame}

% ELENCO PUNTATO
\begin{frame}{Itemization}
    \begin{itemize}
        \item \scriptsize {Uno} % scriptsize è uno dei comandi per la dimensione del testo, in questo caso rimpicciolisce, cerca su internet la lista dei comandi per le dimensioni del testo
        \item \large {Due} % large comando per ingrandire il testo, esiste anche huge
        \item Tre
    \end{itemize}
\end{frame}

% per mettere in GRASSETTO/CORSIVO una parola: o la seleziono e poi clicco in alto come su world, oppure c'è il comando \textit per il corsivo e \textbf per il grassetto e metto la parola che devo modificare tra {}

\section{Formulas}
\begin{frame}{Formulas used}
In questa tesi abbiamo usato la deviazione standard:
\bigskip % mi mette uno spazio tra testo e formula, posso anche usare \smallskip
    \begin{equation} %per i simboli puoi sempre cercare su internet la lista dei comandi e impostazioni varie. \sum è il comando che mi inserisce la sommatoria e il pezzettino dopo sum lo uso per mettergli i limiti, \frac è quello per la divisione/rapporto e metto tra {} numeratore e denominatore, \sqrt invece è per la radice quadrata
        \delta = \sqrt{\frac{\displaystyle\sum_{i=1}^N (x - \mu)^2}{N}} %metto il simbolo \ prima del mu e me lo rende in lettere greche, stessa cosa metto \delta
        %\displaystyle lo uso per mettere i limiti della sommarotia sopra e sotto il simbolo, quindi rende tutto più bello esteticamente
    \end{equation}
\end{frame}

\section{Results}

% inseriamo immagine
\begin{frame}{Archived results} %poi da qui vado nei 3 puntini in alto e faccio insert figure, mi mette in automatico questa roba, elimino la roba che non mi serve come \label e \caption
    \begin{figure}
        \centering
        \includegraphics[width=0.9\linewidth]{graf.png}  % Solitamente è bene usare 0.9 o anche solo .9 perchè si adatta bene a tutto
% aggiungo \raggedright e \raggedleft per spostare a dx o a sx
    \end{figure}
\end{frame}

% inserire due figure una di fianco all'altra: ricopio il pezzo di prima e aggiugno questo
\begin{frame}{Due immagini vicine} 
    \begin{figure}
        \centering
        \includegraphics[width=0.4\linewidth]{graf.png} 
        \pause \includegraphics[width=0.4\linewidth]{graf.png} %ho messo \pause per inserirre un'animazione per l'immagine
    \end{figure}
\end{frame}

% metto quattro immagini vicine, due sopra e due sotto
\begin{frame}{quattro immagini vicine} % anche qui per le immagini posso inserire \pause e farle comparire una alla volta
    \begin{figure}
        \centering
        \includegraphics[width=0.3\linewidth]{graf.png} 
        \includegraphics[width=0.3\linewidth]{graf.png} \\ % doppio backslash per andare a capo
        \includegraphics[width=0.3\linewidth]{graf.png}
        \includegraphics[width=0.3\linewidth]{graf.png}
    \end{figure}
\end{frame}

% per inserire un testo con font courier tipico dei linguaggi di programmazione informatica
\begin{frame}{Courier}
    In this project we made use of the R package
    \texttt{imageRy} %imageRy mi viene scritto con quel font
\end{frame}


\end{document}
